Title: Improving the Quality of Experience of Mobile Internet Applications

Abstract: Mobile Internet applications have become predominant with the prolific increase in smartphone usage. Typically, these applications include Web browsing, video streaming and interactive telephony. One of the performance issues of these applications is users' quality of experience (QoE). The QoE of these applications is influenced by a number of factors such as the network, hardware and software performance. This makes it hard to pinpoint and diagnose the problem and bringing new challenges in providing a good quality of experience (QoE) for these applications. 

Providing a good QoE is important because the users tend to leave the application as soon as they have bad experience. Unfortunately, traditional quality of serivce (QoS) based network resource provisioning is not very correlated with users' QoE and thus it is not efficient. In this work, our goal is to maximize the QoE of Mobile Internet applications by QoE based resource control. Towards this, we study the QoE of three popular applications -- YouTube for streaming, Skype for telephony and Google chrome for Web browsing. 

We first conduct a survey of existing work that make an effort in QoE optimization and hilight the limitations. We then propose a machine (deep) learning based QoE model that is placed at a WiFi access point or cellular base station. The model learns the QoE of applications under diverse network conditions and helps the network administrator in resource provisioning to maximize the QoE. Our model improves the QoE by 40% compared to traditional models with 93% median accuracy. Finally, we perform a detailed study of underlying device performance to understand device side bottlenecks when there are sufficient network resources. This is especially important for users in developing regions. This investigation also enables content providers different strategies for device-specific and/or the network-side QoE optimizations. 
