\section{Introduction}
The massive use of smartphone Internet applications makes their quality of experience (QoE) increasingly critical in our daily life. 
As of the year 2018, 51.12\% of the global web traffic is served to mobile phones while more than 63\% of the video traffic is mobile \cite{statista2}. 
This is majorly due to the growing availability of diverse mobile Internet applications such as mobile browsing, video streaming, interactive telephony and many other social network applications.

However, with the advent of heterogeneous devices, networks and applications between content providers and the end-users, it has become difficult to uncover if a user is having QoE issues and pointing out the root cuase of any issue. 
Apart from server-side issues and Internet delays, the users face several issues on client-side ecosystem -- the client connected network parameters such as delay, throughput and loss, underlying device configuration such as OS and hardware, and the inherent application complexity.

The goal of this work is to improve the QoE of mobile Internet applications by addressing client-side problems in terms of network as well device. 
To achieve this, we i) study the performance of three popular mobile Internet applications -- Web browsing, Video Streaming and Video Telephony, with respect to diverse network and device conditions, ii) identify the limitations of prior work in optimizing network resources based on end-user QoE and, propose a scalable and application independent QoE management method, iii) make an effort to alleviate the device-specific problems for browsing applications.

There is a large body of existing work in optimizing network and device resources to improve mobile Internet QoE \cite{jana2016qoe}. However, the these solutions have two critical limitations: Recent success of using machine learning (ML) in networking enabled several network optimizations for QoE management \cite{jana2016qoe}. For example, network administrators map traditional QoS parameters such as delay, throughput and loss to end-user QoE using ML algorithms and control network resources according the end-user QoE. To achieve this, the prior work collects the QoE information from applications and corresponding network statistics to train the ML model. However, majority of the existing work rely on applications to get the ground-truth, use inappropriate QoE metrics, not scalable across diverse applications (\S\ref{MOTIVATION}).
Most of the network optimizations are designed for desktops especially for the Web \cite{nejati2016depth}. 
The models work well if the root cause of the QoE issues is network alone. This is not the case with mobile devices that vary significantly in terms of OS and hardware.
The model results in underutilized network resources if the network administrator controls the network resources according to a poor QoE from a very low-end device. 
Furthermore, different applications demand different requirements. For example, a Web browsing application is delay sensitive where it requires a quick page load while a video application is bandwdith intesive.
Therefore, the optimzations should consider the impact of device, network as well as the application complexity.

In summary, our contributions and findings are following:

\begin{itemize}
\item We micro benchmark the three popular applications -- browsing, video streaming and telephony with diverse network and device configurations. We find that the behavior these applications significantly differ with device impact while relatively similar with network impact (\S \ref{label:design} and \S \ref{intro}).
\item We propose a scalable and application independent QoE prediction model by addressing the limitation of existing work. We achieve a median 90\% accuracy in studying three video telephony applications -- Skype, FaceTime and Google Hangouts (\S \ref{label:model}).
\item We find that Web browsing is most sensitive to device impact while other applications exploiting specialized hardware accelerators. We offload certain browsing computation onto coprocessors and achieve 18.5\% improvement in speeding up page loads.
\end{itemize}
