%!TEX root = ../main.tex
\begin{abstract}

Mobile Internet applications have become predominant with the prolific increase in smartphone usage. 
Typically, these applications are Web browsing and video applications such as streaming and interactive telephony. 
The problem here is that the applications are influenced by a number of factors such as the network parameters, hardware performance and the inherent application complexity. 
This makes it hard to pinpoint and diagnose the problem, bringing new challenges in providing a good quality of experience (QoE) for these applications.
To this end, we first conduct a systematic survey of existing literature that makes effort to improve the mobile Internet QoE.
We then show the evidence for limitations of prior work in QoE optimizations in terms of network and device performance.
Our contributions of this work are two-fold. First, we propose a machine learning based QoE model that is hooked on top of WiFi access point or a cellular base station. 
The model learns the QoE of applications under diverse network conditions and help the network administrator in network resource provisioning according to the end-user QoE rather than traditional quality of serive (QoS) parameters.
Second, we dig deep into the effect of underlying device performance to understand device side bottlenecks while providing unlimited network resources. 
This investigation brings a variety of strategies to content providers in placing the optimizations at mobile device or the network side which is especially important for the people of developing regions.
Our experimental results shows that the behavior of these applications varies significantly from one another with respect to both network and hardware resources.
We show the performance of these applications by selecting most popular applications such as YouTube for streaming, Skype for telephony and Google chrome for Web browsing. 

\end{abstract}
