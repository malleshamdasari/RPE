\section{Conclusion}
In this work, we studied QoE of popular mobile Internet applications--- Web browsing, video streaming, and telephony applications across diverse network and device conditions.

From the networking-standpoint, we have shown that prior QoE metrics does not scale across diverse video content and depend on application provided APIs. 
We proposed a generic QoE prediction ML model by using content and application independent QoE metrics.
Our model achieves a median accuracy of 90\%.
We are currently extending the work by using deep learning techniques such as CNN and LSTM to capture spatial and temporal features that our handcrafted could not capture. 
In future, we further plan to extend the work to other applications such as video streaming and VR/AR applications.

From the device-standpoint, we have shown that different applications are affected differently because of application specific optimizations.
For example, Web browsing is more sensitive to CPU clock whereas video applications are not. 
This is because video applications are exploiting specialized coprocessors even on low-end mobile devices and are effectively using multi-cores on smartphones.
We offload part of Web browsing computation (Javascript) to DSP and show an improvement of 18\% in page loads time and 4$\times$ reduction energy consumption.


